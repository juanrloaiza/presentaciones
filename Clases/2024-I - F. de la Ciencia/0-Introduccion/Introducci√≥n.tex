% Options for packages loaded elsewhere
\PassOptionsToPackage{unicode}{hyperref}
\PassOptionsToPackage{hyphens}{url}

\documentclass[
  ignorenonframetext,
]{beamer}

  \usepackage{pgfpages}
  \setbeamertemplate{caption}[numbered]
  \setbeamertemplate{caption label separator}{: }
  \setbeamercolor{caption name}{fg=normal text.fg}
  \beamertemplatenavigationsymbolsempty

  %%
  %%% Definition of colors
  %%% Source: https://latexcolor.com/
  \definecolor{blanchedalmond}{rgb}{1.0, 0.92, 0.8}
  \definecolor{blond}{rgb}{0.98, 0.94, 0.75}
  %%% End of definition of colors
  %%

    % Prevent slide breaks in the middle of a paragraph
  \widowpenalties 1 10000
  \raggedbottom
      \setbeamertemplate{part page}{
      \centering
      \begin{beamercolorbox}[sep=16pt,center]{part title}
        \usebeamerfont{part title}\insertpart\par
      \end{beamercolorbox}
    }
    \setbeamertemplate{section page}{
      \centering
      \begin{beamercolorbox}[sep=12pt,center]{part title}
        \usebeamerfont{section title}\insertsection\par
      \end{beamercolorbox}
    }
    \setbeamertemplate{subsection page}{
      \centering
      \begin{beamercolorbox}[sep=8pt,center]{part title}
        \usebeamerfont{subsection title}\insertsubsection\par
      \end{beamercolorbox}
    }
    \AtBeginPart{
      \frame{\partpage}
    }
    \AtBeginSection{
      \ifbibliography
      \else
        \frame{\sectionpage}
      \fi
    }
    \AtBeginSubsection{
      \frame{\subsectionpage}
    }
  

  \usepackage{lmodern}


\usepackage{amssymb,amsmath}
\usepackage{ifxetex,ifluatex}
\ifnum 0\ifxetex 1\fi\ifluatex 1\fi=0 % if pdftex
  \usepackage[T1]{fontenc}
  \usepackage[utf8]{inputenc}
  \usepackage{textcomp} % provide euro and other symbols
\else % if luatex or xetex
  \usepackage{unicode-math}
  \defaultfontfeatures{Scale=MatchLowercase}
  \defaultfontfeatures[\rmfamily]{Ligatures=TeX,Scale=1}
  \setmainfont[]{Linux Libertine}

\fi

\usefonttheme{serif} % use mainfont rather than sansfont for slide text
% Use upquote if available, for straight quotes in verbatim environments
\IfFileExists{upquote.sty}{\usepackage{upquote}}{}


\usepackage{xcolor}
\IfFileExists{xurl.sty}{\usepackage{xurl}}{} % add URL line breaks if available
\usepackage{bookmark}
\usepackage{hyperref}
\hypersetup{
  hidelinks,
}


  
\newif\ifbibliography




  
  
  
  
\setlength{\emergencystretch}{3em} % prevent overfull lines

\usepackage{enumitem}

\providecommand{\tightlist}{%
  \setlength{\itemsep}{0pt}\setlength{\parskip}{0pt}}


    \setcounter{secnumdepth}{-\maxdimen} % remove section numbering
  
%
% When using babel or polyglossia with biblatex, loading csquotes is recommended 
% to ensure that quoted texts are typeset according to the rules of your main language.
%
\usepackage{csquotes}

%
% blockquote
%
\definecolor{blockquote-border}{RGB}{221,221,221}
\definecolor{blockquote-text}{RGB}{89,89,89}


%
% Source Sans Pro as the de­fault font fam­ily
% Source Code Pro for monospace text
%
% 'default' option sets the default 
% font family to Source Sans Pro, not \sfdefault.
%
    
    
  
  
  
    
      \date{}
      

\begin{document}

\hypertarget{quuxe9-es-la-ciencia}{%
\section{¿Qué es la «ciencia»?}\label{quuxe9-es-la-ciencia}}

\begin{frame}{Definición preliminar}
\protect\hypertarget{definiciuxf3n-preliminar}{}
La \textbf{ciencia} es una práctica humana de producción de
conocimiento.

\textless Diagrama de conocimiento vs.~conocimiento científico y otros
conocimientos\textgreater{}
\end{frame}

\begin{frame}{Ejemplares}
\protect\hypertarget{ejemplares}{}
¿Qué ciencias son ejemplares canónicas de «ciencias»?

\begin{columns}[T]
\begin{column}{0.48\textwidth}
\textbf{Ciencias}

\begin{itemize}
\tightlist
\item
  Física
\item
  Biología
\item
  Química
\end{itemize}
\end{column}

\begin{column}{0.48\textwidth}
\textbf{No ciencias}

\begin{itemize}
\tightlist
\item
  Astrología
\item
  Adivinación
\item
  Homeopatía
\end{itemize}
\end{column}
\end{columns}

Hay otras disciplinas que son áreas grises.

\begin{itemize}
\tightlist
\item
  Ciencias sociales
\item
  Psicología
\item
  Humanidades
\end{itemize}
\end{frame}

\begin{frame}{El problema de la demarcación}
\protect\hypertarget{el-problema-de-la-demarcaciuxf3n}{}
La pregunta por distinguir \textbf{ciencia} de \textbf{no-ciencia} (o
\emph{pseudociencia}) es conocido como el \textbf{problema de la
demarcación}.

Si es posible demarcar entre ciencia y no-ciencia, debe haber algún
\textbf{criterio de demarcación}.

Criterios tradicionales de demarcación:

\begin{itemize}
\tightlist
\item
  Método científico
\item
  Verificación, confirmación
\end{itemize}
\end{frame}

\begin{frame}{El problema de la demarcación}
\protect\hypertarget{el-problema-de-la-demarcaciuxf3n-1}{}
\begin{block}{Método científico}
\protect\hypertarget{muxe9todo-cientuxedfico}{}
Se dice que la ciencia se identifica por el uso de un \textbf{método
científico}.

Pregunta -\textgreater{} Hipótesis -\textgreater{} Observación
-\textgreater{} Análisis -\textgreater{} Resultado

Varias disciplinas ``pseudocientíficas'' cumplen con el uso de métodos
similares:

\begin{quote}
Pregunta: ¿Qué rasgos de la personalidad tienen las personas nacidas en
agosto? Hipótesis: Tienden a ser entusiastas y creativos Observación:
Algunas personas nacidas en agosto son entusiastas y creativas Análisis:
La observación concuerda con la hipótesis Resultado: La hipótesis es
verdadera
\end{quote}
\end{block}
\end{frame}

\begin{frame}{El problema de la demarcación}
\protect\hypertarget{el-problema-de-la-demarcaciuxf3n-2}{}
\begin{block}{Método científico}
\protect\hypertarget{muxe9todo-cientuxedfico-1}{}
Adicionalmente, algunos elementos de las ciencias paradigmáticas no
siguen este método.

\begin{quote}
``Todo cuerpo persevera en su estado de reposo o movimiento uniforme y
rectilíneo a no ser que sea obligado a cambiar su estado por fuerzas
impresas sobre él.''
\end{quote}

Es imposible obtener un sistema experimental sin ninguna fuerza.

\begin{itemize}
\tightlist
\item
  Como mínimo, el observador será un cuerpo imprimiendo fuerza sobre el
  objeto en el sistema.
\end{itemize}

Consecuencia: Es imposible confirmar experimentalmente la primera ley de
Newton.
\end{block}
\end{frame}

\begin{frame}{El problema de la demarcación}
\protect\hypertarget{el-problema-de-la-demarcaciuxf3n-3}{}
\begin{block}{Verificación y confirmación}
\protect\hypertarget{verificaciuxf3n-y-confirmaciuxf3n}{}
Podríamos pensar que la ciencia se identifica por solo creer enunciados
\textbf{bien confirmados} (i.e., con buena evidencia en su favor).

Algunos problemas:

\begin{enumerate}
\tightlist
\item
  Toda la evidencia posible confirma hipótesis triviales (e.g., ``Los
  capricornio son caprichosos o no lo son'').
\item
  Con frecuencia creemos hipótesis antes de que haya evidencia en su
  favor (e.g., la relatividad no tenía todavía más evidencia en su favor
  que las teorías anteriores).
\item
  Podemos interpretar la evidencia a favor de casi cualquier hipótesis
  (e.g., epiciclos)
\end{enumerate}
\end{block}
\end{frame}

\begin{frame}{El problema de la demarcación}
\protect\hypertarget{el-problema-de-la-demarcaciuxf3n-4}{}
\begin{block}{¿Dónde está el problema?}
\protect\hypertarget{duxf3nde-estuxe1-el-problema}{}
Con todo, parte del problema parece estar en la relación que hay entre
los \textbf{hechos} y nuestras \textbf{teorías}.

\begin{itemize}
\tightlist
\item
  Las teorías parecen contener expectativas sobre los hechos.
\item
  Los hechos parecen confirmar y falsear teorías.
\item
  Necesitamos confrontar los hechos para saber cuáles teorías son
  verdaderas y cuáles son falsas.
\end{itemize}

Esto invita a algunas preguntas:

\begin{itemize}
\tightlist
\item
  ¿Qué son los \textbf{hechos}?
\item
  ¿Qué son las \textbf{teorías}?
\item
  ¿Cuál es su conexión? ¿Podemos revisar hechos sin teoría?
\end{itemize}
\end{block}
\end{frame}

\end{document}
