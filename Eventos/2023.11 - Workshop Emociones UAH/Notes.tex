\documentclass{article}

\usepackage[
    includemp,
    margin=1in,
    marginparsep=0.4in,
    marginparwidth=1in
    ]{geometry}

\usepackage{fontspec}
\setmainfont{Roboto}

\usepackage[spanish]{babel}
\usepackage{amsmath}
\usepackage{xcolor}

\setlength{\parindent}{0pt}
\setlength{\parskip}{6pt plus 2pt minus 1pt}


\usepackage{marginnote}
\newcommand{\annotation}[1]{\marginnote{\small\color{gray}#1}}

\providecommand{\tightlist}{%
  \setlength{\itemsep}{0pt}\setlength{\parskip}{0pt}}

\usepackage{multicol}

\begin{document}

{\LARGE Anglocentrismo en la ciencia de las emociones}

Juan R. Loaiza -- Universidad Alberto Hurtado

\vspace{0.8em}
\hrule
\vspace{1em}

\section{Introducción}\label{introducciuxf3n}

\subsection{La ciencia de la emoción y los conceptos
anglocéntricos}\label{la-ciencia-de-la-emociuxf3n-y-los-conceptos-anglocuxe9ntricos}

La ciencia de las emociones ha usado tradicionalmente conceptos del
inglés.\annotation{Ekman, 1992; Panksepp, 1998}

\begin{itemize}
\tightlist
\item
  Emociones básicas de Ekman: \emph{joy}, \emph{fear}\ldots{}
\item
  Circuitos básicos de Panksepp: \emph{FEAR}, \emph{RAGE},
  \emph{PLAY}\ldots{}
\end{itemize}

Si existen traducciones de estos términos, la ciencia de las emociones
es universalizable.

\subsection{El problema del
anglocentrismo}\label{el-problema-del-anglocentrismo}

Problema: No todos los términos de emoción son traducibles al
inglés.\annotation{Wierzbicka, 1992; Berger}

\begin{itemize}
\tightlist
\item
  \emph{sadness} \(\implies\) \emph{grust} / \emph{pecal} en el ruso.
\item
  \emph{disgust} \(\implies\) a \emph{x} y \emph{y} en el mandarín.
\end{itemize}

Esto lleva a dos problemas para las ciencias de la
emoción:\annotation{Young, 1990}

\begin{itemize}
\tightlist
\item
  Problemas éticos/políticos: Imperialismo cultural y exclusión
\item
  Problemas epistémicos: Ignorancia de distinciones entre emociones
\end{itemize}

\subsection{Propuesta}\label{propuesta}

Una solución es buscar un \emph{metalenguaje universal} para expresar la
teoría de las emociones.\annotation{Wierzbicka, 1992}

\begin{itemize}
\tightlist
\item
  Buscar \emph{universales semánticos} (US).
\item
  Expresar la teoría en términos de los US.
\end{itemize}

\textbf{Tesis}: La solución del metalenguaje universal no es
satisfactoria.

\begin{itemize}
\tightlist
\item
  No evita los problemas epistémicos.
\item
  Arriesga mantener imperialismos culturales.
\end{itemize}

\subsection{Esquema}\label{esquema}

\begin{enumerate}
\def\labelenumi{\arabic{enumi}.}
\tightlist
\item
  El problema con los conceptos anglocéntricos de emoción
\item
  La búsqueda de un metalenguaje universal
\item
  Problemas del metalenguaje universal
\item
  Propuesta: Universalismo metodológico y relativismo
\end{enumerate}

\section{El problema con los conceptos anglocéntricos de
emoción}\label{el-problema-con-los-conceptos-anglocuxe9ntricos-de-emociuxf3n}

\subsection{Teorías de las emociones
básicas}\label{teoruxedas-de-las-emociones-buxe1sicas}

Una de las teorías dominantes de la emoción es la \emph{Teoría de las
emociones básicas} (BET).

BET: Existe un conjunto limitado de emociones que subyacen a toda la
taxonomía emocional.

\begin{itemize}
\tightlist
\item
  Emociones básicas de Ekman: \emph{joy}, \emph{fear}, \emph{anger},
  \emph{surprise}, \emph{disgust}, \emph{sadness}.
\item
  Circuitos básicos de Panksepp: \emph{FEAR}, \emph{RAGE},
  \emph{PLAY}\ldots{}
\end{itemize}

Estas taxonomías se basan en intuiciones del inglés.

\begin{itemize}
\tightlist
\item
  \emph{Sadness} traduce a dos conceptos diferentes en el ruso:
  \emph{grust} y \emph{pecal}.
\end{itemize}

\subsection{Teorías evaluativas}\label{teoruxedas-evaluativas}

Las teorías evaluativas de la emoción también incurren en este problema.

Teorías evaluativas: Las emociones son, en parte, compromisos de valor.

\begin{quote}
``\emph{fear}, \_anger, and \emph{grief} are categories that come
naturally to people and that seem to have considerable cross-linguistic
generality'' (Ellsworth \& Scherer, 2003, p.~588)
\end{quote}

Wierzbicka anota que, por ejemplo, \emph{grief} y \emph{anger} no se
encuentran en muchos idiomas.

\subsection{El problema de las clases
naturales}\label{el-problema-de-las-clases-naturales}

La filosofía de la ciencia de la emoción ha mostrado que no podemos usar
conceptos cotidianos sin calificación.

Los conceptos cotidianos de emoción no refieren a clases naturales.

Griffiths (1997): Hay al menos tres grupos distintos de emociones en el
habla cotidiana.

\begin{itemize}
\tightlist
\item
  Emociones básicas (programas afectivos)
\item
  Emociones de alto nivel cognitivo
\item
  Pretensiones socialmente sostenidas
\end{itemize}

Barrett (2006): No hay correspondencias neuronales o fisiológicas para
los conceptos cotidianos de emoción.

\subsection{Hacia conceptos científicos de
emoción}\label{hacia-conceptos-cientuxedficos-de-emociuxf3n}

La solución al problema de las clases naturales ha sido separar
lenguajes cotidianos y científicos.

Russell (2009): \ldots{}

Scarantino (2012): \emph{Folk Emotion Project} y \emph{Scientific
Emotion Project}.

El problema es que se acuña el inglés para expresar el lenguaje
científico de la emoción.

\section{La búsqueda de un metalenguaje
universal}\label{la-buxfasqueda-de-un-metalenguaje-universal}

\subsection{El diagnóstico del
anglocentrismo}\label{el-diagnuxf3stico-del-anglocentrismo}

Wierzbicka (1999; 2014) critica las teorías tradicionales por recaer en
conceptos del inglés.

\begin{quote}
It is ethnocentric to think that if the Tahitians don't have a word
corresponding to the English word \emph{sad} (Levy 1973), they must
nonetheless have an innate conceptual category of ``sadness''; or to
assume that in their emotional experience ``sadness'' -for which they
have no name- is nonetheless more salient and more relevant to their
``emotional universe'' than, for example, the feelings of
t\textbackslash=\{o\}iaha or \emph{pe'ape'a}, for which they do have a
name (although English does not). (1999, p.~26)
\end{quote}

\subsection{El diagnóstico del
anglocentrismo}\label{el-diagnuxf3stico-del-anglocentrismo-1}

Levisen () hace eco de las críticas de Wierzbicka, y ofrece la siguiente
definición:

\begin{quote}
Anglocentrism: The tacit practice of (i) taking English-specific
concepts to be neutral, natural, universal, and universally applicable,
and (ii) applying this set of ethnocentric misconceptions to the framing
of research questions and methods, the analysis of data, the
interpretation of results, and the establishment of scholarly discourse
and terminologies, (iii) with an inevitable distortion of the
representation of non-English speakers, non-English linguistic
categories, non-Anglo scholarships, and non-Anglo perspectives on human
life and living. (p.~4)
\end{quote}

\subsection{Dos clases de problemas}\label{dos-clases-de-problemas}

Tenemos entonces dos clases de problemas con el anglocentrismo.

\begin{enumerate}
\def\labelenumi{\arabic{enumi}.}
\tightlist
\item
  Problemas políticos y éticos

  \begin{itemize}
  \tightlist
  \item
    Es injusto aplicar conceptos etnocéntricos como estándar para
    comprender las experiencias de otro grupo.
  \end{itemize}
\item
  Problemas epistemológicos

  \begin{itemize}
  \tightlist
  \item
    Desconocemos de la experiencia de ciertos grupos asumiendo la
    estructura de un grupo dominante.
  \end{itemize}
\end{enumerate}

Podemos analizar estos daños en términos de \emph{imperialismo cultural}
(Young, 1990) e \emph{injusticia epistémica} (Fricker, 2009).

\subsection{Universales semánticos}\label{universales-semuxe1nticos}

Anna Wierzbicka propone usar \emph{universales semánticos} para crear un
\textbf{metalenguaje}.

\begin{itemize}
\tightlist
\item
  Universales semánticos
\end{itemize}

Con estos US, se propone analizar conceptos cotidianos de emoción.
\end{document}